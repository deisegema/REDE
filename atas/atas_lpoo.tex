\documentclass[12pt]{article}
\usepackage[portuges]{babel}
\usepackage{amsmath,amssymb,amsxtra,latexsym}
\bibliographystyle{plain}
\usepackage[latin1]{inputenc}
\usepackage{graphicx}
\usepackage{enumerate}
\usepackage{makeidx}
\usepackage{longtable}
\usepackage{subfig}
\usepackage{multirow}
\usepackage{multicol}
\usepackage{pdfpages}
\usepackage{hyperref}
\pagestyle{plain}
%
%
% ---------------------
% Defini\c{c}\~ao das Margens
% ---------------------
\setlength{\textwidth}{16.0cm} %
\setlength{\textheight}{23.5cm} %
\setlength{\baselineskip}{0.6cm} %
\renewcommand{\baselinestretch}{1.1} %
%\setlength{\evensidemargin}{0.5cm} %
\setlength{\oddsidemargin}{0.00in} %
\setlength{\topmargin}{-1.0in} %
\setlength{\headsep}{0.5in} %
\setlength{\headheight}{0.2in} %

\setlength{\parindent}{1.25cm} %

%\setcounter{page}{1}

\begin{document}


\centerline{
\textit{\Large\textbf{\emph{{Ata das Reuni\~oes de 2013}}}}}

\vspace{1cm}

\section{Maio de 2013}

\subsection{3 de maio}

{\bf{ Palestra:}}

\begin{itemize}
\item Dynamic Control of Infeasibility for Nonlinear Programming - {\textbf{ Abel Soares Siqueira}}

\end{itemize}
{\bf{Rede Colaborativa:}}

\begin{itemize}
\item Automatiza\c{c}\~ao de testes - {\textbf{ Raniere Gaia Costa}}

\item Cria\c{c}\~ao de arquivo execut\'avel - {\textbf {Abel Soares Siqueira}}

\end{itemize}
{\bf{Participantes:}}

\begin{itemize}
\item Abel Soares Siqueira
\item Bruno Henrique Cervelin
\item Daiane Gon\c{c}alves Ferreira
\item Darwin Castillo Huamani
\item Deise Gon\c{c}alves Ferreira
\item Hector Flores
\item Kelly Cadena Madrid
\item Raniere Gaia Costa
\end{itemize}

\subsection{8 de maio}

{\bf Reuni\~ao extraordin\'aria}

\begin{itemize}
\item Discutimos se o manifesto e as atas das reuni\~oes ser\~ao armazenadas em um mesmo reposit\'orio e se ficar\~ao p\'ublicas.
\item Decidimos que ficar\~ao p\'ublicas e armazenadas em um mesmo reposit\'orio.
\item Deise ficou de ajudar o Abel com o manifesto.
\item Discutimos a sistem\'atica dos exerc\'icios, ficou decidido:
\begin{itemize}
\item utlizaremos o formato padr\~ao criando a lista com o ambiente enumerate do latex;
\item As listas ser\~ao organizadas pelo Abel em parceria com o professor que mi\-nis\-trou a aula;
\item ap\'os criar os exerc\'icios o professor cria o arquivo tex e envia para a Daiane;
\item a Daiane dever\'a conferir o arquivo recebido incluir no programa e subir no git, depois enviar para o Raniere que dever\'a colocar no site.
\end{itemize}

\item A Deise dever\'a criar as atas, enviar via email para o grupo do lpoo e subir no git.
\item Ao receberem a ata o Abel e o Raniere atualizam no site.
 
\end{itemize}

{\bf{Participantes:}}

\begin{itemize}
\item Abel Soares Siqueira
\item Daiane Gon\c{c}alves Ferreira
\item Deise Gon\c{c}alves Ferreira
\item Raniere Gaia Costa
\end{itemize}

\subsection{10 de maio}

{\bf Ata da reuni\~ao:}

\begin{itemize}
\item Foram cadastrados 7 novos integrantes no sistema da Rede Colaborativa;
\item Dos inscritos na rede:
\begin{itemize}
\item 6 t\^em interesse em aprender $C/C_{++}$;
\item 4 t\^em interesse em aprender fortran;
\item 2 t\^em interesse em aprender Octave/Matlab;
\end{itemize}
\item Para elaborar os exerc\'icios usaremos o pacote exercise devido aos tags;
\item Na pr\'oxima semana o Abel dar\'a uma aula introdut\'oria sobre $C/C_{++}$;
\item Na pr\'oxima semana o Raniere vai apresentar uma aula b\'asica de git;
\item No dia 7 de junho Hector apresentar\'a uma aula introdut\'oria de Fortran.
\end{itemize}

{\bf Palestra}

\begin{itemize}
\item T\'opicos de Pontos Interiores - {\bf Porfirio Su\~nagua Salgado}
\end{itemize}

{\bf Rede Colaborativa:}

\begin{itemize}
\item Processando Logs com GAWK - {\bf Raniere Gaia Costa}
\item Exemplos do awk - {\bf Abel Soares Siqueira}
\end{itemize}

{\bf{Participantes:}}

\begin{itemize}
\item Abel Soares Siqueira
\item Carlos Rubianez
\item Cecilia Orellana Castro
\item Daiane Gon\c{c}alves Ferreira
\item Deise Gon\c{c}alves Ferreira
\item Denis Cajas
\item Hector Flores
\item Jenny Luzgarda Condori Zamora
\item Manolo Rodriguez Eredia
\item Kelly Cadena Madrid
\item Porfirio Su\~nagua Salgado
\item Raniere Gaia Costa
\end{itemize}

\subsection{15 de maio}

{\bf Reuni\~ao extraordin\'aria}

\begin{itemize}
\item O Abel apresentou ao Raniere o  ASCIIIO que \'e software livre (licença MIT) e o c\'odigo encontra-se em:\\
 https://github.com/sickill/ascii.io e https://github.com/sickill/ascii.io-cli.
\item Este servi\c{c}o de grava\c{c}\~ao do terminal ser\'a utilizado para gravar as aulas da rede.
\end{itemize}

{\bf Pariticpantes}

\begin{itemize}
\item Abel Soares Siqueira
\item Raniere Gaia Costa
\end{itemize}

\subsection{17 de maio}

{\bf Rede Colaborativa:} 

\begin{itemize}
\item Um novo integrante foi cadastrado no sistema da Rede Colaborativa.

\item Aula de introdu\c{c}\~ao ao C - {\bf Abel Soares Siqueira}
\item Aula sobre os comandos b\'asicos do Linux - {\bf Raniere Gaia Costa}
\end{itemize}

{\bf{Participantes:}}

\begin{itemize}
\item Abel Soares Siqueira
\item Bruno Henrique Cervelin
\item Carlos Rubianez
\item Cecilia Orellana Castro
\item Deise Gon\c{c}alves Ferreira
\item Denis Cajas
\item Hector Flores
\item Kelly Cadena Madrid
\item Manolo Rodriguez Eredia
\item Raniere Gaia Costa
\item Ronnei Carlos Teixeira
\end{itemize}



\end{document}